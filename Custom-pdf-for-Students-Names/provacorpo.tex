\pagestyle{empty}
	
\hrule
\begin{multicols}{2}
   \hspace{-4cm}
	\includegraphics[width=\linewidth/2]{logo.jpg}%LOGOTIPO DA INSTITUICAO	
		{\centering
		
		\hspace{-5cm}{\textbf{NOMEDOGAJO}} \\
		\hspace{-5cm}Docente:            Prof. Ph.D. Marcelo da Silva Trindade\\
		\hspace{-5cm}Disciplina:         Geometria Analítica do curso CURSODOGAJO             \\
		\hspace{-5cm}Data: 19 de setembro de 2018                \\
		\hspace{-5cm}Duração:            2 horas         \\
	    }
	
	
\end{multicols}
	
	\hrule
	
	\begin{center}
	
	 	\vspace{2pt} 	
		
		\Large \textbf{Avaliação A1 de Geometria Analítica}
		
	\end{center}
	
	
\begin{snugshade}
	\section*{Instruções para a realização da Avaliação}
\end{snugshade}
{\it 
	{\bf (A)} Cada um dos itens abaixo vale 1,25 ponto, pelo que 8 acertos garantem a nota máxima. {\bf (B)} Resolva cada questão em folhas de rescunho à caneta ou a lápis. (serão estas entregues e agrafadas na prova pelo que a não entrega invalida a avaliação). {\bf (C)} Coloque a resposta final de cada questão OBRIGATORIAMENTE na frente da prova à caneta, no local indicado e arredondada em três casas decimais (quando aplicável). O descumprimento desta norma invalida a questão. {\bf (D)} Está permitido o uso de material de consulta impresso e manuscrito e calculadoras.}

\begin{snugshade}
	\section*{Questões a serem resolvidas}
\end{snugshade}

\begin{itemize}
	\item [(1)] Dados os vetores $\vec{u}=(2,1,-1)$ e $\vec{v}=(1,2,5)$ calcular $||\vec{u}||(\vec{u}\times\vec{v})+(\vec{u}-\vec{v})$.
	\item [(2)] Dados os vetores $\vec{u}=(2,0,1)$ e $\vec{v}=(2,-1,-1)$ calcular o tamanho da projeção: $proj_{\vec{u}}\ \vec{v}$.
	\item [(3)] Dado $\vec{u}=(-3,2,4)$, calcular o valor de $m$ em $\vec{v}=(-9,6,2m^2-2)$ para que $\vec{u}/\hspace{-0.1cm}/\vec{v}$.
	\item [(4)] Dados $\vec{u}=(-2,1,3)$  e $\vec{v}=(1,4,-1)$, calcular os ângulos diretores (em graus) de $3\vec{u}-2\vec{v}$.
	\item [(5)] Obtenha a tangente do ângulo formado entre os vetores $\vec{u}=(0,-1,3)$ e $\vec{v}=(2,1,0)$. 
	\item [(6)] Dados $A=(-3,1,1)$ e $B=(6,4,-2)$ encontre um versor de $\vec{AB}$ com  início em $C=(5,2,-1)$.
	\item [(7)] Encontre um vetor $\vec{w}$ tal que $\vec{w}\perp\vec{u}$ e $\vec{w}\perp\vec{v}$, com $\vec{u}=\vec{i}+\vec{j}-\vec{k}$ e $\vec{v}=3\vec{i}-\vec{j}+2\vec{k}$.
	\item [(8)] Se $\vec{u}\bullet\vec{v}=3$ e o ângulo formado entre $\vec{u}$ e $\vec{v}$ é 60 graus, calcule $||\vec{u}||\cdot||\vec{v}||$.
	\item [(9)] Se $-(\vec{u}\times\vec{v})=\vec{i}+3\vec{k}-2\vec{j}$ calcule $m$ em $\vec{w}=(1,2,m-1)$ para que $(\vec{v}\times\vec{u})\perp\vec{w}$. 
\end{itemize}
	
\begin{snugshade}
	\section*{Solução final de cada questão}
\end{snugshade}
\noindent\\
1:\Bigg(\hspace{5cm}\Bigg)
2:\Bigg(\hspace{5cm}\Bigg)
3:\Bigg(\hspace{5cm}\Bigg)
4:\Bigg(\hspace{5cm}\Bigg)
5:\Bigg(\hspace{5cm}\Bigg)
6:\Bigg(\hspace{5cm}\Bigg)
7:\Bigg(\hspace{5cm}\Bigg)
8:\Bigg(\hspace{5cm}\Bigg)
9:\Bigg(\hspace{5cm}\Bigg)
\\
\vspace{0.5cm}
\hrule
\newpage
